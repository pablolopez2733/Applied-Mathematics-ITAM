% Options for packages loaded elsewhere
\PassOptionsToPackage{unicode}{hyperref}
\PassOptionsToPackage{hyphens}{url}
%
\documentclass[
]{article}
\usepackage{lmodern}
\usepackage{amssymb,amsmath}
\usepackage{ifxetex,ifluatex}
\ifnum 0\ifxetex 1\fi\ifluatex 1\fi=0 % if pdftex
  \usepackage[T1]{fontenc}
  \usepackage[utf8]{inputenc}
  \usepackage{textcomp} % provide euro and other symbols
\else % if luatex or xetex
  \usepackage{unicode-math}
  \defaultfontfeatures{Scale=MatchLowercase}
  \defaultfontfeatures[\rmfamily]{Ligatures=TeX,Scale=1}
\fi
% Use upquote if available, for straight quotes in verbatim environments
\IfFileExists{upquote.sty}{\usepackage{upquote}}{}
\IfFileExists{microtype.sty}{% use microtype if available
  \usepackage[]{microtype}
  \UseMicrotypeSet[protrusion]{basicmath} % disable protrusion for tt fonts
}{}
\makeatletter
\@ifundefined{KOMAClassName}{% if non-KOMA class
  \IfFileExists{parskip.sty}{%
    \usepackage{parskip}
  }{% else
    \setlength{\parindent}{0pt}
    \setlength{\parskip}{6pt plus 2pt minus 1pt}}
}{% if KOMA class
  \KOMAoptions{parskip=half}}
\makeatother
\usepackage{xcolor}
\IfFileExists{xurl.sty}{\usepackage{xurl}}{} % add URL line breaks if available
\IfFileExists{bookmark.sty}{\usepackage{bookmark}}{\usepackage{hyperref}}
\hypersetup{
  pdftitle={Proyecto 2},
  pdfauthor={Fernando Stein Vallarta 165455; Pablo López Landeros 178863; Manuel García Garduño 162136},
  hidelinks,
  pdfcreator={LaTeX via pandoc}}
\urlstyle{same} % disable monospaced font for URLs
\usepackage[margin=1in]{geometry}
\usepackage{color}
\usepackage{fancyvrb}
\newcommand{\VerbBar}{|}
\newcommand{\VERB}{\Verb[commandchars=\\\{\}]}
\DefineVerbatimEnvironment{Highlighting}{Verbatim}{commandchars=\\\{\}}
% Add ',fontsize=\small' for more characters per line
\usepackage{framed}
\definecolor{shadecolor}{RGB}{248,248,248}
\newenvironment{Shaded}{\begin{snugshade}}{\end{snugshade}}
\newcommand{\AlertTok}[1]{\textcolor[rgb]{0.94,0.16,0.16}{#1}}
\newcommand{\AnnotationTok}[1]{\textcolor[rgb]{0.56,0.35,0.01}{\textbf{\textit{#1}}}}
\newcommand{\AttributeTok}[1]{\textcolor[rgb]{0.77,0.63,0.00}{#1}}
\newcommand{\BaseNTok}[1]{\textcolor[rgb]{0.00,0.00,0.81}{#1}}
\newcommand{\BuiltInTok}[1]{#1}
\newcommand{\CharTok}[1]{\textcolor[rgb]{0.31,0.60,0.02}{#1}}
\newcommand{\CommentTok}[1]{\textcolor[rgb]{0.56,0.35,0.01}{\textit{#1}}}
\newcommand{\CommentVarTok}[1]{\textcolor[rgb]{0.56,0.35,0.01}{\textbf{\textit{#1}}}}
\newcommand{\ConstantTok}[1]{\textcolor[rgb]{0.00,0.00,0.00}{#1}}
\newcommand{\ControlFlowTok}[1]{\textcolor[rgb]{0.13,0.29,0.53}{\textbf{#1}}}
\newcommand{\DataTypeTok}[1]{\textcolor[rgb]{0.13,0.29,0.53}{#1}}
\newcommand{\DecValTok}[1]{\textcolor[rgb]{0.00,0.00,0.81}{#1}}
\newcommand{\DocumentationTok}[1]{\textcolor[rgb]{0.56,0.35,0.01}{\textbf{\textit{#1}}}}
\newcommand{\ErrorTok}[1]{\textcolor[rgb]{0.64,0.00,0.00}{\textbf{#1}}}
\newcommand{\ExtensionTok}[1]{#1}
\newcommand{\FloatTok}[1]{\textcolor[rgb]{0.00,0.00,0.81}{#1}}
\newcommand{\FunctionTok}[1]{\textcolor[rgb]{0.00,0.00,0.00}{#1}}
\newcommand{\ImportTok}[1]{#1}
\newcommand{\InformationTok}[1]{\textcolor[rgb]{0.56,0.35,0.01}{\textbf{\textit{#1}}}}
\newcommand{\KeywordTok}[1]{\textcolor[rgb]{0.13,0.29,0.53}{\textbf{#1}}}
\newcommand{\NormalTok}[1]{#1}
\newcommand{\OperatorTok}[1]{\textcolor[rgb]{0.81,0.36,0.00}{\textbf{#1}}}
\newcommand{\OtherTok}[1]{\textcolor[rgb]{0.56,0.35,0.01}{#1}}
\newcommand{\PreprocessorTok}[1]{\textcolor[rgb]{0.56,0.35,0.01}{\textit{#1}}}
\newcommand{\RegionMarkerTok}[1]{#1}
\newcommand{\SpecialCharTok}[1]{\textcolor[rgb]{0.00,0.00,0.00}{#1}}
\newcommand{\SpecialStringTok}[1]{\textcolor[rgb]{0.31,0.60,0.02}{#1}}
\newcommand{\StringTok}[1]{\textcolor[rgb]{0.31,0.60,0.02}{#1}}
\newcommand{\VariableTok}[1]{\textcolor[rgb]{0.00,0.00,0.00}{#1}}
\newcommand{\VerbatimStringTok}[1]{\textcolor[rgb]{0.31,0.60,0.02}{#1}}
\newcommand{\WarningTok}[1]{\textcolor[rgb]{0.56,0.35,0.01}{\textbf{\textit{#1}}}}
\usepackage{graphicx,grffile}
\makeatletter
\def\maxwidth{\ifdim\Gin@nat@width>\linewidth\linewidth\else\Gin@nat@width\fi}
\def\maxheight{\ifdim\Gin@nat@height>\textheight\textheight\else\Gin@nat@height\fi}
\makeatother
% Scale images if necessary, so that they will not overflow the page
% margins by default, and it is still possible to overwrite the defaults
% using explicit options in \includegraphics[width, height, ...]{}
\setkeys{Gin}{width=\maxwidth,height=\maxheight,keepaspectratio}
% Set default figure placement to htbp
\makeatletter
\def\fps@figure{htbp}
\makeatother
\setlength{\emergencystretch}{3em} % prevent overfull lines
\providecommand{\tightlist}{%
  \setlength{\itemsep}{0pt}\setlength{\parskip}{0pt}}
\setcounter{secnumdepth}{-\maxdimen} % remove section numbering
\usepackage{booktabs}
\usepackage{longtable}
\usepackage{array}
\usepackage{multirow}
\usepackage{wrapfig}
\usepackage{float}
\usepackage{colortbl}
\usepackage{pdflscape}
\usepackage{tabu}
\usepackage{threeparttable}
\usepackage{threeparttablex}
\usepackage[normalem]{ulem}
\usepackage{makecell}
\usepackage{xcolor}

\title{Proyecto 2}
\author{Fernando Stein Vallarta 165455 \and Pablo López Landeros 178863 \and Manuel García Garduño 162136}
\date{}

\begin{document}
\maketitle

\hypertarget{introducciuxf3n}{%
\subsection{Introducción}\label{introducciuxf3n}}

Durante los últimos 10 años, las redes sociales han crecido a un ritmo
descomunal y parece que cada vez toman mayor relevancia en varios
aspectos de nuestras vidas. Esta tendencia ha dado pie a estudiar a
través de la estadística, todo lo que fluye a través de estas redes,
esto con el objetivo de poder transformar los datos en información, la
información en conocimiento y el conocimiento en \emph{insights}. Es por
esto que ya hoy en día existen gran cantidad de APIs y código
``prefabricado'' que nos permite facilmente minar los datos que se
encuentran en los servidores de Twitter, Facebook y/o Instagram.

Para un primer acercamiento a estas herramientas se plantea este
proyecto. Consiste en utilizar los conceptos de muestreo visto en clase
y el paquete de R \emph{rtweet} para estudiar la estructura de Twitter.
Concretamente, enfocamos el estudio en extraer y analizar el contenido
de tweets que involucren o discutan al presidente de los Estados Unidos:
\textbf{Donald Trump}.

\begin{Shaded}
\begin{Highlighting}[]
\CommentTok{#Importamos librerías:}
\KeywordTok{rm}\NormalTok{(}\DataTypeTok{list=} \KeywordTok{ls}\NormalTok{())}

\CommentTok{#Importamos librerias}

\CommentTok{# 0. Importamos Librerias}
\KeywordTok{library}\NormalTok{(cowplot)}
\KeywordTok{library}\NormalTok{(dplyr) }
\KeywordTok{library}\NormalTok{(glue)}
\KeywordTok{library}\NormalTok{(ggplot2)}
\KeywordTok{library}\NormalTok{(ggmap)}
\KeywordTok{library}\NormalTok{(ggcorrplot)}
\KeywordTok{library}\NormalTok{(jsonlite)}
\KeywordTok{library}\NormalTok{(kableExtra)}
\KeywordTok{library}\NormalTok{(knitr)}
\KeywordTok{library}\NormalTok{(ks)}
\KeywordTok{library}\NormalTok{(leaflet)}
\KeywordTok{library}\NormalTok{(lubridate)}
\KeywordTok{library}\NormalTok{(mapsapi)}
\KeywordTok{library}\NormalTok{(moments) }
\KeywordTok{library}\NormalTok{(rtweet)}
\KeywordTok{library}\NormalTok{(rjson)}
\KeywordTok{library}\NormalTok{(RCurl)}
\KeywordTok{library}\NormalTok{(stringr)}
\KeywordTok{library}\NormalTok{(tidytext)}
\KeywordTok{library}\NormalTok{(tidyverse) }
\KeywordTok{library}\NormalTok{(tmaptools)}
\end{Highlighting}
\end{Shaded}

\hypertarget{extracciuxf3n-de-los-tweets-muestra}{%
\subsubsection{Extracción de los Tweets
muestra}\label{extracciuxf3n-de-los-tweets-muestra}}

Este es el código que originalmente utilizamos para extraer las muestras
pero R markdown no puede hacer el render. Por lo que hubo que correr
estas líneas por separado en un script y posteriormente leer los
archivos RDS generados.

\begin{Shaded}
\begin{Highlighting}[]
\CommentTok{#Parte 1: Importamos los tweets y los convertimos en una base de datos. }
\CommentTok{#Según el link https://github.com/ropensci/rtweet/issues/356 existe un error en la configuración }
\CommentTok{#de rTweet cuando usamos files de tipo json, luego entonces utilizamos el fix proporcionado en el }
\CommentTok{#siguiente link: https://gist.github.com/JBGruber/dee4c44e7d38d537426f57ba1e4f84ab}

\CommentTok{#El fix consiste en la función recover_stream, la cual se construye en las siguientes líneas:}

\CommentTok{#' Recovers Twitter damaged stream data (JSON file) into parsed data frame.}
\CommentTok{#'}
\CommentTok{#' @param path Character, name of JSON file with data collected by}
\CommentTok{#'   \textbackslash{}code\{\textbackslash{}link\{stream_tweets\}\}.}
\CommentTok{#' @param dir Character, name of a directory where intermediate files are}
\CommentTok{#'   stored.}
\CommentTok{#' @param verbose Logical, should progress be displayed?}
\CommentTok{#'}
\CommentTok{#' @family stream tweets}
\NormalTok{recover_stream <-}\StringTok{ }\ControlFlowTok{function}\NormalTok{(path, }\DataTypeTok{dir =} \OtherTok{NULL}\NormalTok{, }\DataTypeTok{verbose =} \OtherTok{TRUE}\NormalTok{) \{}
  
  \CommentTok{# read file and split to tweets}
\NormalTok{  lines <-}\StringTok{ }\KeywordTok{readChar}\NormalTok{(path, }\KeywordTok{file.info}\NormalTok{(path)}\OperatorTok{$}\NormalTok{size, }\DataTypeTok{useBytes =} \OtherTok{TRUE}\NormalTok{)}
\NormalTok{  tweets <-}\StringTok{ }\NormalTok{stringi}\OperatorTok{::}\KeywordTok{stri_split_fixed}\NormalTok{(lines, }\StringTok{"}\CharTok{\textbackslash{}n}\StringTok{\{"}\NormalTok{)[[}\DecValTok{1}\NormalTok{]]}
\NormalTok{  tweets[}\OperatorTok{-}\DecValTok{1}\NormalTok{] <-}\StringTok{ }\KeywordTok{paste0}\NormalTok{(}\StringTok{"\{"}\NormalTok{, tweets[}\OperatorTok{-}\DecValTok{1}\NormalTok{])}
\NormalTok{  tweets <-}\StringTok{ }\NormalTok{tweets[}\OperatorTok{!}\NormalTok{(tweets }\OperatorTok{==}\StringTok{ ""} \OperatorTok{|}\StringTok{ }\NormalTok{tweets }\OperatorTok{==}\StringTok{ "\{"}\NormalTok{)]}
  
  \CommentTok{# remove misbehaving characters}
\NormalTok{  tweets <-}\StringTok{ }\KeywordTok{gsub}\NormalTok{(}\StringTok{"}\CharTok{\textbackslash{}r}\StringTok{"}\NormalTok{, }\StringTok{""}\NormalTok{, tweets, }\DataTypeTok{fixed =} \OtherTok{TRUE}\NormalTok{)}
\NormalTok{  tweets <-}\StringTok{ }\KeywordTok{gsub}\NormalTok{(}\StringTok{"}\CharTok{\textbackslash{}n}\StringTok{"}\NormalTok{, }\StringTok{""}\NormalTok{, tweets, }\DataTypeTok{fixed =} \OtherTok{TRUE}\NormalTok{)}
  
  \CommentTok{# write tweets to disk and try to read them in individually}
  \ControlFlowTok{if}\NormalTok{ (}\KeywordTok{is.null}\NormalTok{(dir)) \{}
\NormalTok{    dir <-}\StringTok{ }\KeywordTok{paste0}\NormalTok{(}\KeywordTok{tempdir}\NormalTok{(), }\StringTok{"/tweets/"}\NormalTok{)}
    \KeywordTok{dir.create}\NormalTok{(dir, }\DataTypeTok{showWarnings =} \OtherTok{FALSE}\NormalTok{)}
\NormalTok{  \}}
  
  \ControlFlowTok{if}\NormalTok{ (verbose) \{}
\NormalTok{    pb <-}\StringTok{ }\NormalTok{progress}\OperatorTok{::}\NormalTok{progress_bar}\OperatorTok{$}\KeywordTok{new}\NormalTok{(}
      \DataTypeTok{format =} \StringTok{"Processing tweets [:bar] :percent, :eta remaining"}\NormalTok{,}
      \DataTypeTok{total =} \KeywordTok{length}\NormalTok{(tweets), }\DataTypeTok{clear =} \OtherTok{FALSE}
\NormalTok{    )}
\NormalTok{    pb}\OperatorTok{$}\KeywordTok{tick}\NormalTok{(}\DecValTok{0}\NormalTok{)}
\NormalTok{  \}}
  
\NormalTok{  tweets_l <-}\StringTok{ }\KeywordTok{lapply}\NormalTok{(tweets, }\ControlFlowTok{function}\NormalTok{(t) \{}
\NormalTok{    pb}\OperatorTok{$}\KeywordTok{tick}\NormalTok{()}
\NormalTok{    id <-}\StringTok{ }\KeywordTok{unlist}\NormalTok{(stringi}\OperatorTok{::}\KeywordTok{stri_extract_first_regex}\NormalTok{(t, }\StringTok{"(?<=id}\CharTok{\textbackslash{}"}\StringTok{:)}\CharTok{\textbackslash{}\textbackslash{}}\StringTok{d+(?=,)"}\NormalTok{))[}\DecValTok{1}\NormalTok{]}
\NormalTok{    f <-}\StringTok{ }\KeywordTok{paste0}\NormalTok{(dir, id, }\StringTok{".json"}\NormalTok{)}
    \KeywordTok{writeLines}\NormalTok{(t, f, }\DataTypeTok{useBytes =} \OtherTok{TRUE}\NormalTok{)}
\NormalTok{    out <-}\StringTok{ }\KeywordTok{tryCatch}\NormalTok{(rtweet}\OperatorTok{::}\KeywordTok{parse_stream}\NormalTok{(f),}
                    \DataTypeTok{error =} \ControlFlowTok{function}\NormalTok{(e) \{\})}
    \ControlFlowTok{if}\NormalTok{ (}\StringTok{"tbl_df"} \OperatorTok\StringTok{ }\KeywordTok{class}\NormalTok{(out)) \{}
      \KeywordTok{return}\NormalTok{(out)}
\NormalTok{    \} }\ControlFlowTok{else}\NormalTok{ \{}
      \KeywordTok{return}\NormalTok{(id)}
\NormalTok{    \}}
\NormalTok{  \})}
  
  \CommentTok{# test which ones failed}
\NormalTok{  test <-}\StringTok{ }\KeywordTok{vapply}\NormalTok{(tweets_l, is.character, }\DataTypeTok{FUN.VALUE =} \KeywordTok{logical}\NormalTok{(1L))}
\NormalTok{  bad_files <-}\StringTok{ }\KeywordTok{unlist}\NormalTok{(tweets_l[test])}
  
  \CommentTok{# Let user decide what to do}
  \ControlFlowTok{if}\NormalTok{ (}\KeywordTok{length}\NormalTok{(bad_files) }\OperatorTok{>}\StringTok{ }\DecValTok{0}\NormalTok{) \{}
    \KeywordTok{message}\NormalTok{(}\StringTok{"There were "}\NormalTok{, }\KeywordTok{length}\NormalTok{(bad_files),}
            \StringTok{" tweets with problems. Should they be copied to your working directory?"}\NormalTok{)}
\NormalTok{    sel <-}\StringTok{ }\KeywordTok{menu}\NormalTok{(}\KeywordTok{c}\NormalTok{(}\StringTok{"no"}\NormalTok{, }\StringTok{"yes"}\NormalTok{, }\StringTok{"copy a list with status_ids"}\NormalTok{))}
    \ControlFlowTok{if}\NormalTok{ (sel }\OperatorTok{==}\StringTok{ }\DecValTok{2}\NormalTok{) \{}
      \KeywordTok{dir.create}\NormalTok{(}\KeywordTok{paste0}\NormalTok{(}\KeywordTok{getwd}\NormalTok{(), }\StringTok{"/broken_tweets/"}\NormalTok{), }\DataTypeTok{showWarnings =} \OtherTok{FALSE}\NormalTok{)}
      \KeywordTok{file.copy}\NormalTok{(}
        \DataTypeTok{from =} \KeywordTok{paste0}\NormalTok{(dir, bad_files, }\StringTok{".json"}\NormalTok{),}
        \DataTypeTok{to =} \KeywordTok{paste0}\NormalTok{(}\KeywordTok{getwd}\NormalTok{(), }\StringTok{"/broken_tweets/"}\NormalTok{, bad_files, }\StringTok{".json"}\NormalTok{)}
\NormalTok{      )}
\NormalTok{    \} }\ControlFlowTok{else} \ControlFlowTok{if}\NormalTok{ (sel }\OperatorTok{==}\StringTok{ }\DecValTok{3}\NormalTok{) \{}
      \KeywordTok{writeLines}\NormalTok{(bad_files, }\StringTok{"broken_tweets.txt"}\NormalTok{)}
\NormalTok{    \}}
\NormalTok{  \}}
  
  \CommentTok{# clean up}
  \KeywordTok{unlink}\NormalTok{(dir, }\DataTypeTok{recursive =} \OtherTok{TRUE}\NormalTok{)}
  
  \CommentTok{# return good tweets}
  \KeywordTok{return}\NormalTok{(dplyr}\OperatorTok{::}\KeywordTok{bind_rows}\NormalTok{(tweets_l[}\OperatorTok{!}\NormalTok{test]))}
\NormalTok{\}}

\CommentTok{# 1. Minería de Datos}
\CommentTok{# Después de emplear el fix, comenzamos con la minería de datos/tweets}
\CommentTok{# Notemos que tomaremos 2 muestras, pues esto nos servirá para realizar captura y recaptura}
\CommentTok{#Mineria de Datos}
\NormalTok{file.name1 <-}\StringTok{ "data_tw.json"}
\NormalTok{file.name2 <-}\StringTok{ "data_tw2.json"}
\NormalTok{time.ellapsed<-}\StringTok{ }\DecValTok{600}

\CommentTok{#MUESTRA 1}
\KeywordTok{stream_tweets}\NormalTok{(      }\StringTok{"Trump"}\NormalTok{,}
                    \DataTypeTok{parse=}\OtherTok{FALSE}\NormalTok{,}
                    \DataTypeTok{file_name=}\NormalTok{file.name1,}
                    \DataTypeTok{language =} \StringTok{"en"}\NormalTok{, }
                    \DataTypeTok{timeout =}\NormalTok{ time.ellapsed}
\NormalTok{)}

\NormalTok{data_}\DecValTok{1}\NormalTok{<-}\KeywordTok{recover_stream}\NormalTok{(file.name1)}
\NormalTok{data_}\DecValTok{1}\NormalTok{<-}\StringTok{ }\KeywordTok{as.data.frame}\NormalTok{(data_}\DecValTok{1}\NormalTok{)}

\CommentTok{#MUESTRA 2}
\KeywordTok{stream_tweets}\NormalTok{(      }\StringTok{"Trump"}\NormalTok{,}
                    \DataTypeTok{parse=}\OtherTok{FALSE}\NormalTok{,}
                    \DataTypeTok{file_name=}\NormalTok{file.name2,}
                    \DataTypeTok{language =} \StringTok{"en"}\NormalTok{, }
                    \DataTypeTok{timeout =}\NormalTok{ time.ellapsed}
\NormalTok{)}

\NormalTok{data_}\DecValTok{2}\NormalTok{<-}\KeywordTok{recover_stream}\NormalTok{(file.name2)}
\NormalTok{data_}\DecValTok{2}\NormalTok{ <-}\StringTok{ }\KeywordTok{as.data.frame}\NormalTok{(data_}\DecValTok{2}\NormalTok{)}
\end{Highlighting}
\end{Shaded}

\hypertarget{lectura-de-los-rds}{%
\paragraph{Lectura de los RDS}\label{lectura-de-los-rds}}

\begin{Shaded}
\begin{Highlighting}[]
\NormalTok{data_}\DecValTok{1}\NormalTok{ <-}\StringTok{ }\KeywordTok{as.data.frame}\NormalTok{(}\KeywordTok{readRDS}\NormalTok{(}\DataTypeTok{file =} \StringTok{'data_1.rds'}\NormalTok{))}
\NormalTok{data_}\DecValTok{2}\NormalTok{ <-}\StringTok{ }\KeywordTok{as.data.frame}\NormalTok{(}\KeywordTok{readRDS}\NormalTok{(}\DataTypeTok{file =} \StringTok{'data_2.rds'}\NormalTok{))}

\CommentTok{#Para ambos data.frames nos quedamos con las variables de relevancia}
\NormalTok{data1<-}\StringTok{ }\NormalTok{data_}\DecValTok{1} \OperatorTok\StringTok{ }\KeywordTok{select}\NormalTok{(user_id, status_id, screen_name, text,verified)}
\NormalTok{data2<-data_}\DecValTok{2} \OperatorTok\StringTok{ }\KeywordTok{select}\NormalTok{(user_id, status_id, screen_name, text, verified)}

\CommentTok{#Para realizar captura y recaptura juntaremos las dos bases de datos y buscaremos los textos}
\CommentTok{# únicos para obtener nuestra "k"}
\NormalTok{r<-}\KeywordTok{rbind}\NormalTok{(data1,data2) }
\NormalTok{r.len <-}\StringTok{ }\NormalTok{r}\OperatorTok\KeywordTok{count}\NormalTok{()}
\NormalTok{unicos <-}\StringTok{ }\NormalTok{r }\OperatorTok\StringTok{ }\KeywordTok{distinct}\NormalTok{(text)}\OperatorTok\KeywordTok{tally}\NormalTok{()}\OperatorTok\KeywordTok{as.numeric}\NormalTok{() }
\NormalTok{repetidos<-}\StringTok{ }\NormalTok{r.len }\OperatorTok{-}\StringTok{ }\NormalTok{unicos}
\end{Highlighting}
\end{Shaded}

\hypertarget{estimaciuxf3n-del-total-de-cuentas-verificadas}{%
\subsubsection{Estimación del total de cuentas
verificadas}\label{estimaciuxf3n-del-total-de-cuentas-verificadas}}

En Twitter, una cuenta verificada aparece con una palomita azul a un
lado del nombre de usuario. Esto permite a la gente saber que esa cuenta
de interés público es auténtica. Por lo tanto, esto no solo es una
mejora estética sino una manera de darle un poco más respaldo a lo que
se escriba en la red social. Es por eso que nos interesamos en estimar
un total de cuentas verificadas que se involucran en hablar sobre Donald
Trump.

Para estimar el total y sus intervalos de confianza, necesitamos primero
la \(N\). La estimaremos adaptando el método de captura recaptura
propuesto por Lohr en el capítulo 13 del libro \emph{Sampling: Design
and Analysis}. Utilizaremos la fórmula: \[ \hat{N} = \frac{n_1 n_2}{m}\]
Para adaptar el método, capturamos tweets durante 5 mins dos veces.
Obteniendo así \(n_1\) com el tamaño de la primer captura. \(n_2\) el
tamaño de la segunda captura y \(m\) como la cantidad de usuarios que
aparecían en ambas capturas.

Una vez teniendo un estimador de \(N\). utilizaremos el estimador para
el total visto en clase:
\[\hat{t} = N \bar{x_s} = N \frac{1}{n_1} \sum_{i=1}^{n_1} x_i\]

Con intervalos de confianza:
\[\hat{t} \pm Z_{1-\frac{\alpha}{2}} \sqrt{\hat{V(\hat{t}) }}\] donde:
\[ \hat{V(\hat{t})} = N^2 \frac{1-\frac{n_1}{N}}{n_1} S^2_{x,s}\] y a su
vez:

\[ S^2_{x,s} = \frac{1}{n_1-1} \sum_{k=1}^{n_1} (x_k - \bar{x_s} )^2\]
\#\#\#\#\#\# Valor esperado y Varianza del estimador:

\[E[\hat{t}] = E [  \sum_{k=1}^{N} \frac{x_k}{\pi_k} \mathbb{I}_S  ] =  \sum_{k=1}^{N} \frac{x_k}{\pi_k} E [\mathbb{I}_S (x_k)] =  \sum_{k=1}^{N} \frac{x_k}{\pi_k} \pi_k = t\]

Es decir, nuestro estimador \(\hat{t}\) es insesgado. Obtengamos
entonces un estimador del total con su respectivo intervalo de confianza
al 99\%:

\begin{Shaded}
\begin{Highlighting}[]
\CommentTok{#Obtenemos nuestra N}
\NormalTok{n1 <-}\StringTok{ }\KeywordTok{nrow}\NormalTok{(data1)}
\NormalTok{n2 <-}\StringTok{ }\KeywordTok{nrow}\NormalTok{(data2)}
\NormalTok{m <-}\StringTok{ }\KeywordTok{as.numeric}\NormalTok{(repetidos)}
\NormalTok{N <-}\StringTok{ }\NormalTok{(n1 }\OperatorTok{*}\StringTok{ }\NormalTok{n2)}\OperatorTok{/}\NormalTok{m }\CommentTok{#Utilizando la formula de captura y recaptura}
\NormalTok{N <-}\StringTok{ }\KeywordTok{as.numeric}\NormalTok{(N) }\OperatorTok{*}\DecValTok{100} \CommentTok{#Notamos que es el 1% de la muestra}

\CommentTok{# 2. Buscamos encontrar el número de cuentas verificadas que hablan acerca de Trump}

\NormalTok{verified <-}\StringTok{ }\NormalTok{data_}\DecValTok{1} \OperatorTok\StringTok{ }\KeywordTok{select}\NormalTok{(user_id,verified)}
\NormalTok{trus <-}\StringTok{ }\KeywordTok{length}\NormalTok{ (verified}\OperatorTok{$}\NormalTok{verified[verified}\OperatorTok{$}\NormalTok{verified}\OperatorTok{==}\OtherTok{TRUE}\NormalTok{])}

\NormalTok{cols <-}\StringTok{ }\KeywordTok{sapply}\NormalTok{(verified, is.logical)}
\NormalTok{verified[,cols] <-}\StringTok{ }\KeywordTok{lapply}\NormalTok{(verified[,cols], as.numeric)}
\end{Highlighting}
\end{Shaded}

\begin{verbatim}
## Warning in `[<-.data.frame`(`*tmp*`, , cols, value = list(0, 0, 0, 0, 0, :
## provided 282 variables to replace 1 variables
\end{verbatim}

\begin{Shaded}
\begin{Highlighting}[]
\NormalTok{x_s <-}\StringTok{ }\KeywordTok{mean}\NormalTok{(verified}\OperatorTok{$}\NormalTok{verified)}
\NormalTok{t_hat <-}\StringTok{ }\NormalTok{N}\OperatorTok{*}\NormalTok{(}\DecValTok{1}\OperatorTok{/}\NormalTok{n1)}\OperatorTok{*}\NormalTok{(}\KeywordTok{sum}\NormalTok{(verified}\OperatorTok{$}\NormalTok{verified))}\OperatorTok{*}\DecValTok{100}
\NormalTok{alpha <-}\StringTok{ }\FloatTok{.01}
\NormalTok{z <-}\StringTok{ }\KeywordTok{qnorm}\NormalTok{(}\DecValTok{1}\OperatorTok{-}\NormalTok{alpha}\OperatorTok{/}\DecValTok{2}\NormalTok{)}
\NormalTok{term <-}\StringTok{ }\KeywordTok{rep}\NormalTok{(}\DecValTok{0}\NormalTok{,n1)}
\CommentTok{#estimar sx}
\ControlFlowTok{for}\NormalTok{ (i }\ControlFlowTok{in} \DecValTok{1}\OperatorTok{:}\NormalTok{n1) \{}
  
\NormalTok{  term[i]=(verified}\OperatorTok{$}\NormalTok{verified[i]}\OperatorTok{-}\NormalTok{x_s)}\OperatorTok{^}\DecValTok{2}
\NormalTok{\}}

\NormalTok{s_x <-}\StringTok{ }\NormalTok{((}\DecValTok{1}\OperatorTok{/}\NormalTok{(n1}\DecValTok{-1}\NormalTok{))}\OperatorTok{*}\NormalTok{(}\KeywordTok{sum}\NormalTok{(term))}\OperatorTok{*}\DecValTok{100}\NormalTok{)}

\NormalTok{V_hat <-}\StringTok{ }\NormalTok{N}\OperatorTok{^}\DecValTok{2} \OperatorTok{*}\StringTok{ }\NormalTok{((}\DecValTok{1}\OperatorTok{-}\NormalTok{n1}\OperatorTok{/}\NormalTok{N)}\OperatorTok{/}\NormalTok{n1 )}\OperatorTok{*}\StringTok{ }\NormalTok{s_x}

\CommentTok{#Intervalo de confianza al 99% del total de cuentas verificadas que hablan de Trump}

\NormalTok{IC_ver <-}\StringTok{ }\KeywordTok{c}\NormalTok{(t_hat}\OperatorTok{-}\NormalTok{z}\OperatorTok{*}\KeywordTok{sqrt}\NormalTok{(V_hat),t_hat }\OperatorTok{+}\StringTok{ }\NormalTok{z}\OperatorTok{*}\KeywordTok{sqrt}\NormalTok{(V_hat))}
\NormalTok{ciLow <-}\StringTok{ }\KeywordTok{round}\NormalTok{(IC_ver[}\DecValTok{1}\NormalTok{])}
\NormalTok{ciHigh <-}\StringTok{ }\KeywordTok{round}\NormalTok{(IC_ver[}\DecValTok{2}\NormalTok{]) }
\KeywordTok{paste0}\NormalTok{(}\StringTok{"Se estiman "}\NormalTok{, }\KeywordTok{round}\NormalTok{(t_hat) ,}\StringTok{" cuentas verificadas tweeteando sobre Trump con intervalo de confianza ["}\NormalTok{,}
\NormalTok{       ciLow, }\StringTok{","}\NormalTok{, ciHigh,}\StringTok{"] del 99%"}\NormalTok{)}
\end{Highlighting}
\end{Shaded}

\begin{verbatim}
## [1] "Se estiman 0 cuentas verificadas tweeteando sobre Trump con intervalo de confianza [0,0] del 99%"
\end{verbatim}

Para darnos una idea de la proporción de cuentas verificadas, hagamos
una gráfica de barras que muestre el total de cuentas verificadas y no
verificadas en la muestra.

\begin{Shaded}
\begin{Highlighting}[]
\CommentTok{#Realizamos una gráfica }
\NormalTok{df <-}\StringTok{ }\KeywordTok{data.frame}\NormalTok{(}\DataTypeTok{cuentas=}\KeywordTok{c}\NormalTok{(}\StringTok{"Verificadas"}\NormalTok{, }\StringTok{"No Verificadas"}\NormalTok{),}
                 \DataTypeTok{cantidad_tweets=}\KeywordTok{c}\NormalTok{(trus,n1}\OperatorTok{-}\NormalTok{trus))}
\NormalTok{p<-}\KeywordTok{ggplot}\NormalTok{(}\DataTypeTok{data=}\NormalTok{df, }\KeywordTok{aes}\NormalTok{(}\DataTypeTok{x=}\NormalTok{cuentas, }\DataTypeTok{y=}\NormalTok{cantidad_tweets)) }\OperatorTok{+}
\StringTok{  }\KeywordTok{geom_bar}\NormalTok{(}\DataTypeTok{stat=}\StringTok{"identity"}\NormalTok{,}\DataTypeTok{fill=}\StringTok{"lightblue"}\NormalTok{)}
\NormalTok{p}
\end{Highlighting}
\end{Shaded}

\includegraphics{ProyectoSocialMedia_files/figure-latex/unnamed-chunk-5-1.pdf}

\hypertarget{estimar-la-diferencia-promedio-entre-seguidos-y-seguidores-de-los-usarios-que-hablan-acerca-de-trump}{%
\subsubsection{Estimar la diferencia promedio entre seguidos y
seguidores de los usarios que hablan acerca de
Trump}\label{estimar-la-diferencia-promedio-entre-seguidos-y-seguidores-de-los-usarios-que-hablan-acerca-de-trump}}

(i.e.~Followers - Following).

\begin{Shaded}
\begin{Highlighting}[]
\CommentTok{# 3. Estimar la diferencia promedio entre seguidos y seguidores de los usarios que hablan acerca de Trump}
\CommentTok{#(i.e. Followers - Following).}
\NormalTok{muestra.usuarios  <-}\StringTok{ }\KeywordTok{users_data}\NormalTok{(data_}\DecValTok{1}\NormalTok{)}

\CommentTok{#Usamos una base de datos más sencilla que nos sirva}
\NormalTok{user_data <-}\StringTok{ }\KeywordTok{data.frame}\NormalTok{(}\StringTok{"user_id"}\NormalTok{ =}\StringTok{ }\NormalTok{muestra.usuarios}\OperatorTok{$}\NormalTok{user_id,}
                        \StringTok{"Followers"}\NormalTok{ =}\StringTok{ }\NormalTok{muestra.usuarios}\OperatorTok{$}\NormalTok{followers_count,}
                        \StringTok{"Following"}\NormalTok{ =}\StringTok{ }\NormalTok{muestra.usuarios}\OperatorTok{$}\NormalTok{friends_count,}
                        \StringTok{"Diferencia"}\NormalTok{ =}\StringTok{ }\NormalTok{muestra.usuarios}\OperatorTok{$}\NormalTok{followers_count}\OperatorTok{-}\NormalTok{muestra.usuarios}\OperatorTok{$}\NormalTok{friends_count)}
\end{Highlighting}
\end{Shaded}

Notemos que la diferencia en datos puede ser muy grande, de ahí que para
que nuestro análisis sea mejor, tennemos que eliminar los ``outliers'':

\begin{Shaded}
\begin{Highlighting}[]
\NormalTok{Q          <-}\StringTok{ }\KeywordTok{quantile}\NormalTok{(user_data}\OperatorTok{$}\NormalTok{Diferencia, }\DataTypeTok{probs=}\KeywordTok{c}\NormalTok{(.}\DecValTok{25}\NormalTok{, }\FloatTok{.75}\NormalTok{), }\DataTypeTok{na.rm =} \OtherTok{FALSE}\NormalTok{)}
\NormalTok{iqr        <-}\StringTok{ }\KeywordTok{IQR}\NormalTok{(user_data}\OperatorTok{$}\NormalTok{Diferencia)}
\NormalTok{eliminated <-}\StringTok{ }\KeywordTok{subset}\NormalTok{(user_data, user_data}\OperatorTok{$}\NormalTok{Diferencia }\OperatorTok{>}\StringTok{ }\NormalTok{(Q[}\DecValTok{1}\NormalTok{] }\OperatorTok{-}\StringTok{ }\FloatTok{1.5}\OperatorTok{*}\NormalTok{iqr) }\OperatorTok{&}\StringTok{ }\NormalTok{user_data}\OperatorTok{$}\NormalTok{Diferencia }\OperatorTok{<}\StringTok{ }\NormalTok{(Q[}\DecValTok{2}\NormalTok{]}\OperatorTok{+}\FloatTok{1.5}\OperatorTok{*}\NormalTok{iqr))}
\end{Highlighting}
\end{Shaded}

Sea Theta el promedio de la cantidad Followers-Following de todos los
nodos en el universo de twitter. Entonces el estimador de máxima
verosimilitud, insesgado y consistente del parámetro Theta está dado por
el promedio de las observaciones en nuestra muestra.

\begin{Shaded}
\begin{Highlighting}[]
\NormalTok{x.barra <-}\StringTok{ }\KeywordTok{mean}\NormalTok{(eliminated}\OperatorTok{$}\NormalTok{Diferencia)}
\end{Highlighting}
\end{Shaded}

Partimos del estimador puntual que previamente hemos definido para crear
un intervalo de confianza al nivel 99\% para el parámetro Theta:

\begin{Shaded}
\begin{Highlighting}[]
\CommentTok{#Partimos del estimador puntual que previamente hemos definido para crear un }
\CommentTok{#intervalo de confianza al nivel 99% para el parámetro Theta.}
\NormalTok{a            <-}\StringTok{ }\KeywordTok{qnorm}\NormalTok{(}\DecValTok{1}\OperatorTok{-}\NormalTok{alpha}\OperatorTok{/}\DecValTok{2}\NormalTok{)}
\NormalTok{desv.muestra <-}\StringTok{ }\KeywordTok{sd}\NormalTok{(eliminated}\OperatorTok{$}\NormalTok{Diferencia)}
\end{Highlighting}
\end{Shaded}

Realizamos un esquema para visualizar bien los datos:

\begin{Shaded}
\begin{Highlighting}[]
\NormalTok{n <-}\StringTok{ }\KeywordTok{length}\NormalTok{(eliminated}\OperatorTok{$}\NormalTok{Diferencia)}
\NormalTok{vec <-}\StringTok{ }\KeywordTok{seq}\NormalTok{(}\DecValTok{1}\OperatorTok{:}\NormalTok{n)}
\KeywordTok{ggplot}\NormalTok{(eliminated, }\KeywordTok{aes}\NormalTok{(}\DataTypeTok{x =}\NormalTok{ vec, }\DataTypeTok{y =}\NormalTok{ Diferencia))}\OperatorTok{+}
\StringTok{  }\KeywordTok{geom_bar}\NormalTok{(}\DataTypeTok{stat =} \StringTok{'identity'}\NormalTok{, }\DataTypeTok{fill =} \StringTok{'blue4'}\NormalTok{)}\OperatorTok{+}
\StringTok{  }\KeywordTok{geom_hline}\NormalTok{(}\DataTypeTok{yintercept =}\NormalTok{ x.barra, }\DataTypeTok{color =} \StringTok{'firebrick'}\NormalTok{)}
\end{Highlighting}
\end{Shaded}

\includegraphics{ProyectoSocialMedia_files/figure-latex/unnamed-chunk-10-1.pdf}
Ahora tenemos los elementos suficientes para construir nuestro intervalo
de confianza al nivel 99\%:

\begin{Shaded}
\begin{Highlighting}[]
\NormalTok{limite.inferior <-}\StringTok{ }\NormalTok{x.barra }\OperatorTok{-}\StringTok{ }\NormalTok{a}\OperatorTok{*}\NormalTok{desv.muestra}
\NormalTok{limite.superior <-}\StringTok{ }\NormalTok{x.barra }\OperatorTok{+}\StringTok{ }\NormalTok{a}\OperatorTok{*}\NormalTok{desv.muestra}
\KeywordTok{paste0}\NormalTok{( }\StringTok{"IC: ["}\NormalTok{,}
\NormalTok{       limite.inferior, }\StringTok{","}\NormalTok{, limite.superior,}\StringTok{"]"}\NormalTok{)}
\end{Highlighting}
\end{Shaded}

\begin{verbatim}
## [1] "IC: [-1402.87939742142,762.846339570176]"
\end{verbatim}

\hypertarget{estimaciuxf3n-de-la-longitud-promedio-de-un-post}{%
\subsubsection{Estimación de la longitud promedio de un
post}\label{estimaciuxf3n-de-la-longitud-promedio-de-un-post}}

Una de las preguntas asignadas se refiere a el tamaño promedio de los
tweets. De donde en primera instancia debemos de obtener la longitud de
los tweets.

\begin{Shaded}
\begin{Highlighting}[]
\NormalTok{len<-}\KeywordTok{nchar}\NormalTok{(data1}\OperatorTok{$}\NormalTok{text, }\DataTypeTok{type =} \StringTok{"chars"}\NormalTok{, }\DataTypeTok{allowNA =} \OtherTok{FALSE}\NormalTok{, }\DataTypeTok{keepNA =} \OtherTok{NA}\NormalTok{)}
\NormalTok{data1}\OperatorTok{$}\NormalTok{len<-len}
\end{Highlighting}
\end{Shaded}

Ahora bien, nosotros tomamos una muestra de tweets utilizando el paquete
de rtweet. Sin embargo, nos interesa conocer el tamaño promedio de tweet
de la población. De ahí que necesitamos \(\bar x_U\), la media
poblacional. Se propone la media muestral como estimador de la muestra
poblacional.

\begin{enumerate}
\def\labelenumi{\arabic{enumi}.}
\tightlist
\item
  Demostracion de que la media muestral es un estimador insesgado:
  \begin{align} \mathrm E(\bar x_s)&= \frac{1}{n}\,\mathrm E{\left(\sum_{i=1}^n x_i\right)}\\&= \frac{1}{n}\,\mathrm E(t_i)\\ &= \frac{1}{n}{\left(\frac{1}{N \atopwithdelims () n}\,\sum_{i=1}^{N \\atopwithdelims () n}t_i\right)}\\ &= \frac{1}{n}\left(\frac{1}{N \atopwithdelims () n}\,\sum_{i=1}^{N \atopwithdelims () n}\left(\sum x_i\right)\right)\\ &= \frac{1}{n}\left(\frac{1}{N \atopwithdelims () n}\,{{N-1\atopwithdelims ()se n-1}\sum_{i=1}^N x_i}\right)\\ &=\frac{1}{N}\sum_{i=1}^{N}y_i\\ &= \overline x_u\;.\end{align}
\end{enumerate}

Luego entonces, la media muestral es un estimador insesgado de la media
poblacional.

Ahora bien, si quisieramos un intervalo de confianza de la media debemos
notar que no conocemos \(\sigma^2\) o la varianza poblacional.

De la teoría de probabilidad sabemos que si \(Z \sim N(0,1)\) and
\(W \sim \chi^2(s)\), entonces la variable
\(Y = \frac{Z}{\sqrt{W/\nu}}\) sigue una distribución \(t\) con \(s\)
grados de libertad.

Luego entonces definimos a Z = \$
\frac{\sqrt n(\bar x_S − \bar x_U)}{\sigma}\$ la cual se distribuye
normal estándar. De igual forma, definimos
\(W = \frac{(n-1)S^2}{\sigma^2}\) sigue una distribución \(\chi^2(s)\)
con \(n-1\) grados de libertad. De ahí que
\(Y = \frac{\bar x_S - \bar x_U}{\frac{S}{\sqrt n}}\) sigue una
distribución \(t\) con \(n-1\) grados de libertad.

De ahí que un intervalo de confianza para \(\bar x_U\) esta dado por
\(\bar x_S \pm t_{\tiny (n-1, 1-\frac{\alpha}{2})}\frac{\sqrt n}{S}\)

Utilizando lo anteriormente encontrado, tenemos los siguientes
resultados al insertar el código en R.

\begin{Shaded}
\begin{Highlighting}[]
\NormalTok{len.bar<-}\StringTok{ }\KeywordTok{mean}\NormalTok{(data1}\OperatorTok{$}\NormalTok{len) }
\NormalTok{n<-}\KeywordTok{length}\NormalTok{(data1}\OperatorTok{$}\NormalTok{len)}
\NormalTok{s<-}\KeywordTok{sd}\NormalTok{(len)}
\end{Highlighting}
\end{Shaded}

Con su respectivo intervalo de confianza dado por:

\begin{Shaded}
\begin{Highlighting}[]
\NormalTok{talpha<-}\KeywordTok{qt}\NormalTok{(}\DecValTok{1}\OperatorTok{-}\NormalTok{alpha}\OperatorTok{/}\DecValTok{2}\NormalTok{,n}\DecValTok{-1}\NormalTok{)}
\NormalTok{ic <-}\StringTok{ }\KeywordTok{round}\NormalTok{(}\KeywordTok{c}\NormalTok{(len.bar }\OperatorTok{-}\StringTok{ }\NormalTok{talpha}\OperatorTok{*}\NormalTok{s}\OperatorTok{/}\KeywordTok{sqrt}\NormalTok{(n), len.bar }\OperatorTok{+}\StringTok{ }\NormalTok{talpha}\OperatorTok{*}\NormalTok{s}\OperatorTok{/}\KeywordTok{sqrt}\NormalTok{(n)), }\DecValTok{2}\NormalTok{)}
\KeywordTok{paste0}\NormalTok{(}\StringTok{"IC: ["}\NormalTok{,}
\NormalTok{       ic[}\DecValTok{1}\NormalTok{], }\StringTok{","}\NormalTok{, ic[}\DecValTok{2}\NormalTok{],}\StringTok{"]"}\NormalTok{)}
\end{Highlighting}
\end{Shaded}

\begin{verbatim}
## [1] "IC: [149.6,176.48]"
\end{verbatim}

\hypertarget{sentiment-analysis}{%
\subsubsection{Sentiment Analysis}\label{sentiment-analysis}}

Para el siguiente punto, buscamos explicar las características
relacionadas con los tweets que se escriben acerca de Donald Trump. En
concreto, queremos saber si los posts que se escriben hablan de forma
positiva, neutral o negativa acerca de Donald Trump. Para esto
necesitamos realizar algo llamado ``Sentiment Analysis'' o Análisis de
Sentimientos en español. El análisis de sentimientos es una técnica de
aprendizaje de maquina que detecta la polaridad dentro de un texto, es
decir, trata de describir las emociones que el texto presenta. Para
esto, utilizaremos los paquetes tidyverse y tidytext junto con el
diccionario ``bing'' realizado por Bing Liu que clasifica las palabras
en positivas y negativas.

\begin{Shaded}
\begin{Highlighting}[]
\NormalTok{dropwords <-}\StringTok{ }\KeywordTok{c}\NormalTok{(}\StringTok{"trump"}\NormalTok{) }\CommentTok{# La palabra "Trump" tiene una conotación en el lexicon}

\NormalTok{sentiment<-data1 }\OperatorTok\StringTok{ }
\StringTok{  }\KeywordTok{unnest_tokens}\NormalTok{(word, text, }\DataTypeTok{token =} \StringTok{'words'}\NormalTok{) }\OperatorTok\StringTok{ }
\StringTok{  }\KeywordTok{left_join}\NormalTok{(}\KeywordTok{get_sentiments}\NormalTok{(}\StringTok{"bing"}\NormalTok{) }\OperatorTok\StringTok{ }\KeywordTok{filter}\NormalTok{(}\OperatorTok{!}\NormalTok{word }\OperatorTok\StringTok{ }\NormalTok{dropwords)) }\OperatorTok\StringTok{ }\CommentTok{# Obtiene las palabras de sentimiento}
\StringTok{  }\KeywordTok{mutate}\NormalTok{(}\DataTypeTok{sentiment =} \KeywordTok{if_else}\NormalTok{(sentiment }\OperatorTok{==}\StringTok{ 'positive'}\NormalTok{, }\DecValTok{1}\NormalTok{, }\DecValTok{-1}\NormalTok{, }\DataTypeTok{missing =} \DecValTok{0}\NormalTok{)) }\OperatorTok\StringTok{ }\CommentTok{# Asigna valores a las palabras de sentimiento }
\StringTok{  }\KeywordTok{group_by}\NormalTok{(user_id) }\OperatorTok\StringTok{ }\CommentTok{#Agrupamos las palabras por clave}
\StringTok{  }\KeywordTok{summarise}\NormalTok{(}\DataTypeTok{sentiment =} \KeywordTok{sum}\NormalTok{(sentiment))}
\end{Highlighting}
\end{Shaded}

\begin{verbatim}
## Joining, by = "word"
\end{verbatim}

\begin{verbatim}
## `summarise()` ungrouping output (override with `.groups` argument)
\end{verbatim}

\begin{Shaded}
\begin{Highlighting}[]
\CommentTok{#Agrupamos por tipo de sentimiento del tweet}

\NormalTok{n.positivo<-}\KeywordTok{subset}\NormalTok{(sentiment,sentiment}\OperatorTok{>}\DecValTok{0}\NormalTok{)}\OperatorTok\KeywordTok{count}\NormalTok{()}
\NormalTok{n.positivo}\OperatorTok{$}\NormalTok{class<-}\KeywordTok{c}\NormalTok{(}\StringTok{"Positivos"}\NormalTok{)}
\NormalTok{n.neutral<-}\KeywordTok{subset}\NormalTok{(sentiment,sentiment}\OperatorTok{==}\DecValTok{0}\NormalTok{)}\OperatorTok\KeywordTok{count}\NormalTok{()}
\NormalTok{n.neutral}\OperatorTok{$}\NormalTok{class<-}\KeywordTok{c}\NormalTok{(}\StringTok{"Neutrales"}\NormalTok{)}
\NormalTok{n.negativo<-}\KeywordTok{subset}\NormalTok{(sentiment,sentiment}\OperatorTok{<}\DecValTok{0}\NormalTok{)}\OperatorTok\KeywordTok{count}\NormalTok{()}
\NormalTok{n.negativo}\OperatorTok{$}\NormalTok{class<-}\KeywordTok{c}\NormalTok{(}\StringTok{"Negativos"}\NormalTok{)}

\NormalTok{n<-}\StringTok{ }\KeywordTok{as.numeric}\NormalTok{(n.positivo}\OperatorTok{$}\NormalTok{n }\OperatorTok{+}\StringTok{ }\NormalTok{n.neutral}\OperatorTok{$}\NormalTok{n }\OperatorTok{+}\StringTok{ }\NormalTok{n.negativo}\OperatorTok{$}\NormalTok{n)}

\NormalTok{count.sentiment<-}\KeywordTok{rbind}\NormalTok{(n.positivo,n.neutral,n.negativo)}

\CommentTok{#Realizamos una tabla y un plot }
\KeywordTok{kable}\NormalTok{(count.sentiment, }\DataTypeTok{booktabs =}\NormalTok{ T) }\OperatorTok\StringTok{ }\KeywordTok{kable_styling}\NormalTok{(}\DataTypeTok{latex_options =} \StringTok{"striped"}\NormalTok{)}
\end{Highlighting}
\end{Shaded}

\begin{table}[H]
\centering
\begin{tabular}{rl}
\toprule
n & class\\
\midrule
\rowcolor{gray!6}  66 & Positivos\\
102 & Neutrales\\
\rowcolor{gray!6}  114 & Negativos\\
\bottomrule
\end{tabular}
\end{table}

\begin{Shaded}
\begin{Highlighting}[]
\KeywordTok{ggplot}\NormalTok{(count.sentiment)}\OperatorTok{+}
\StringTok{  }\KeywordTok{geom_col}\NormalTok{(}\KeywordTok{aes}\NormalTok{(}\DataTypeTok{x =}\NormalTok{ class, }\DataTypeTok{y =}\NormalTok{ n), }\DataTypeTok{color=}\StringTok{'skyblue'}\NormalTok{, }\DataTypeTok{fill=}\StringTok{'steelblue'}\NormalTok{) }\OperatorTok{+}\StringTok{ }
\StringTok{  }\KeywordTok{theme_bw}\NormalTok{() }\OperatorTok{+}\StringTok{ }
\StringTok{  }\KeywordTok{labs}\NormalTok{(}
    \DataTypeTok{title =} \StringTok{"Conteo de Sentimientos de Tweets de Trump"}\NormalTok{,}
    \DataTypeTok{x =} \StringTok{"Tipo de Sentimiento"}\NormalTok{,}
    \DataTypeTok{y =} \StringTok{"Conteo"}
\NormalTok{  ) }\OperatorTok{+}
\StringTok{  }\KeywordTok{theme}\NormalTok{(}\DataTypeTok{axis.text.x=}\KeywordTok{element_text}\NormalTok{(}\DataTypeTok{angle=}\DecValTok{45}\NormalTok{, }\DataTypeTok{hjust=}\DecValTok{1}\NormalTok{))}
\end{Highlighting}
\end{Shaded}

\includegraphics{ProyectoSocialMedia_files/figure-latex/unnamed-chunk-15-1.pdf}

Utilizando lo descrito anteriormente obtenemos los siguientes
resultados:

Ahora bien, notemos que únicamente tenemos una muestra. Pero notemos que
se trata de una muestra estratificada, donde cada estrato es la postura
de cada tweet. Luego entonces, utilizamos la \(N\) obtenida previamente
mediante captura y recaptura. Ahora bien, para obtener el total
poblacional utilizamos la fórmula obtenida en clases dada por
\(n_h = n \frac{N_h}{N}\). Aquí buscamos \(N_h\), de ahí que realizamos
el despeje dado por: \(N_h = N \frac{n_h}{n}\). Con lo anterior
obtenemos los siguientes resultados.

\begin{Shaded}
\begin{Highlighting}[]
\NormalTok{N.positivo <-}\StringTok{ }\KeywordTok{as.numeric}\NormalTok{(n.positivo}\OperatorTok{$}\NormalTok{n)}\OperatorTok{/}\NormalTok{n }\OperatorTok{*}\StringTok{ }\NormalTok{N}
\NormalTok{N.negativo <-}\StringTok{ }\KeywordTok{as.numeric}\NormalTok{(n.negativo}\OperatorTok{$}\NormalTok{n)}\OperatorTok{/}\NormalTok{n }\OperatorTok{*}\StringTok{ }\NormalTok{N}
\NormalTok{N.neutral  <-}\StringTok{ }\KeywordTok{as.numeric}\NormalTok{(n.neutral}\OperatorTok{$}\NormalTok{n)}\OperatorTok{/}\NormalTok{n }\OperatorTok{*}\StringTok{ }\NormalTok{N}
\end{Highlighting}
\end{Shaded}

\hypertarget{estimar-seguidores-de-trump-fuera-de-eu}{%
\subsubsection{Estimar seguidores de Trump fuera de
EU}\label{estimar-seguidores-de-trump-fuera-de-eu}}

Obtenemos una muestra aleatoria de 5,000 followers de Trump: De nuevo,
tuvimos problemas para utilizar el paquete rtweet desde R Markdown por
lo que presentamos el código que usamos para extraer tweets y luego lo
leemos como un rds.

\begin{Shaded}
\begin{Highlighting}[]
\CommentTok{#}
\NormalTok{muestra_aux       <-}\StringTok{ }\KeywordTok{get_followers}\NormalTok{(}\StringTok{"realDonaldTrump"}\NormalTok{, }\DataTypeTok{n =} \DecValTok{5000}\NormalTok{)}
\NormalTok{muestra_aux1      <-}\StringTok{ }\KeywordTok{lookup_users}\NormalTok{(muestra_aux}\OperatorTok{$}\NormalTok{user_id)}
\NormalTok{muestra_followers <-}\StringTok{ }\KeywordTok{users_data}\NormalTok{(muestra_aux1)}
\KeywordTok{saveRDS}\NormalTok{(muestra_followers,}\DataTypeTok{file =} \StringTok{"muestraTrump.rds"}\NormalTok{)}
\end{Highlighting}
\end{Shaded}

\begin{Shaded}
\begin{Highlighting}[]
\NormalTok{muestra_followers <-}\StringTok{ }\KeywordTok{as.data.frame}\NormalTok{(}\KeywordTok{readRDS}\NormalTok{(}\StringTok{'muestraTrump.rds'}\NormalTok{))}
\CommentTok{#Aplicamos la función users_data para obtener la información de su ubicación}


\CommentTok{#Eliminamos de nuestra base de datos aquellas entradas que no tienen especificada}
\CommentTok{#una ubicación}
\NormalTok{muestra_followers <-}\StringTok{ }\NormalTok{muestra_followers}\OperatorTok\KeywordTok{filter}\NormalTok{(location }\OperatorTok{!=}\StringTok{ ""}\NormalTok{)}

\CommentTok{#Obtenemos el tamaño real de nuestra muestra}
\NormalTok{n_muestra         <-}\StringTok{ }\KeywordTok{length}\NormalTok{(muestra_followers}\OperatorTok{$}\NormalTok{user_id)}

\CommentTok{#Ahora nos quedamos con los usuarios que tweetean desde Estados Unidos}
\NormalTok{muestra_followers <-}\StringTok{ }\NormalTok{muestra_followers}\OperatorTok\KeywordTok{filter}\NormalTok{(}\KeywordTok{str_detect}\NormalTok{(location, }\StringTok{"USA"}\NormalTok{)}\OperatorTok{|}
\StringTok{                                                  }\KeywordTok{str_detect}\NormalTok{(location, }\StringTok{"United States"}\NormalTok{)}\OperatorTok{|}
\StringTok{                                                  }\KeywordTok{str_detect}\NormalTok{(location, }\StringTok{", AL"}\NormalTok{)}\OperatorTok{|}
\StringTok{                                                  }\KeywordTok{str_detect}\NormalTok{(location, }\StringTok{", AK"}\NormalTok{)}\OperatorTok{|}
\StringTok{                                                  }\KeywordTok{str_detect}\NormalTok{(location, }\StringTok{", AZ"}\NormalTok{)}\OperatorTok{|}
\StringTok{                                                  }\KeywordTok{str_detect}\NormalTok{(location, }\StringTok{", AR"}\NormalTok{)}\OperatorTok{|}
\StringTok{                                                  }\KeywordTok{str_detect}\NormalTok{(location, }\StringTok{", CA"}\NormalTok{)}\OperatorTok{|}
\StringTok{                                                  }\KeywordTok{str_detect}\NormalTok{(location, }\StringTok{", CO"}\NormalTok{)}\OperatorTok{|}
\StringTok{                                                  }\KeywordTok{str_detect}\NormalTok{(location, }\StringTok{", CT"}\NormalTok{)}\OperatorTok{|}
\StringTok{                                                  }\KeywordTok{str_detect}\NormalTok{(location, }\StringTok{", DE"}\NormalTok{)}\OperatorTok{|}
\StringTok{                                                  }\KeywordTok{str_detect}\NormalTok{(location, }\StringTok{", FL"}\NormalTok{)}\OperatorTok{|}
\StringTok{                                                  }\KeywordTok{str_detect}\NormalTok{(location, }\StringTok{", GA"}\NormalTok{)}\OperatorTok{|}
\StringTok{                                                  }\KeywordTok{str_detect}\NormalTok{(location, }\StringTok{", HI"}\NormalTok{)}\OperatorTok{|}
\StringTok{                                                  }\KeywordTok{str_detect}\NormalTok{(location, }\StringTok{", ID"}\NormalTok{)}\OperatorTok{|}
\StringTok{                                                  }\KeywordTok{str_detect}\NormalTok{(location, }\StringTok{", IL"}\NormalTok{)}\OperatorTok{|}
\StringTok{                                                  }\KeywordTok{str_detect}\NormalTok{(location, }\StringTok{", IN"}\NormalTok{)}\OperatorTok{|}
\StringTok{                                                  }\KeywordTok{str_detect}\NormalTok{(location, }\StringTok{", IA"}\NormalTok{)}\OperatorTok{|}
\StringTok{                                                  }\KeywordTok{str_detect}\NormalTok{(location, }\StringTok{", KS"}\NormalTok{)}\OperatorTok{|}
\StringTok{                                                  }\KeywordTok{str_detect}\NormalTok{(location, }\StringTok{", KY"}\NormalTok{)}\OperatorTok{|}
\StringTok{                                                  }\KeywordTok{str_detect}\NormalTok{(location, }\StringTok{", LA"}\NormalTok{)}\OperatorTok{|}
\StringTok{                                                  }\KeywordTok{str_detect}\NormalTok{(location, }\StringTok{", ME"}\NormalTok{)}\OperatorTok{|}
\StringTok{                                                  }\KeywordTok{str_detect}\NormalTok{(location, }\StringTok{", MD"}\NormalTok{)}\OperatorTok{|}
\StringTok{                                                  }\KeywordTok{str_detect}\NormalTok{(location, }\StringTok{", MA"}\NormalTok{)}\OperatorTok{|}
\StringTok{                                                  }\KeywordTok{str_detect}\NormalTok{(location, }\StringTok{", MI"}\NormalTok{)}\OperatorTok{|}
\StringTok{                                                  }\KeywordTok{str_detect}\NormalTok{(location, }\StringTok{", MN"}\NormalTok{)}\OperatorTok{|}
\StringTok{                                                  }\KeywordTok{str_detect}\NormalTok{(location, }\StringTok{", MS"}\NormalTok{)}\OperatorTok{|}
\StringTok{                                                  }\KeywordTok{str_detect}\NormalTok{(location, }\StringTok{", MO"}\NormalTok{)}\OperatorTok{|}
\StringTok{                                                  }\KeywordTok{str_detect}\NormalTok{(location, }\StringTok{", MT"}\NormalTok{)}\OperatorTok{|}
\StringTok{                                                  }\KeywordTok{str_detect}\NormalTok{(location, }\StringTok{", NE"}\NormalTok{)}\OperatorTok{|}
\StringTok{                                                  }\KeywordTok{str_detect}\NormalTok{(location, }\StringTok{", NV"}\NormalTok{)}\OperatorTok{|}
\StringTok{                                                  }\KeywordTok{str_detect}\NormalTok{(location, }\StringTok{", NH"}\NormalTok{)}\OperatorTok{|}
\StringTok{                                                  }\KeywordTok{str_detect}\NormalTok{(location, }\StringTok{", NJ"}\NormalTok{)}\OperatorTok{|}
\StringTok{                                                  }\KeywordTok{str_detect}\NormalTok{(location, }\StringTok{", NM"}\NormalTok{)}\OperatorTok{|}
\StringTok{                                                  }\KeywordTok{str_detect}\NormalTok{(location, }\StringTok{", NY"}\NormalTok{)}\OperatorTok{|}
\StringTok{                                                  }\KeywordTok{str_detect}\NormalTok{(location, }\StringTok{", NC"}\NormalTok{)}\OperatorTok{|}
\StringTok{                                                  }\KeywordTok{str_detect}\NormalTok{(location, }\StringTok{", ND"}\NormalTok{)}\OperatorTok{|}
\StringTok{                                                  }\KeywordTok{str_detect}\NormalTok{(location, }\StringTok{", OH"}\NormalTok{)}\OperatorTok{|}
\StringTok{                                                  }\KeywordTok{str_detect}\NormalTok{(location, }\StringTok{", OK"}\NormalTok{)}\OperatorTok{|}
\StringTok{                                                  }\KeywordTok{str_detect}\NormalTok{(location, }\StringTok{", OR"}\NormalTok{)}\OperatorTok{|}
\StringTok{                                                  }\KeywordTok{str_detect}\NormalTok{(location, }\StringTok{", PA"}\NormalTok{)}\OperatorTok{|}
\StringTok{                                                  }\KeywordTok{str_detect}\NormalTok{(location, }\StringTok{", RI"}\NormalTok{)}\OperatorTok{|}
\StringTok{                                                  }\KeywordTok{str_detect}\NormalTok{(location, }\StringTok{", SC"}\NormalTok{)}\OperatorTok{|}
\StringTok{                                                  }\KeywordTok{str_detect}\NormalTok{(location, }\StringTok{", SD"}\NormalTok{)}\OperatorTok{|}
\StringTok{                                                  }\KeywordTok{str_detect}\NormalTok{(location, }\StringTok{", TN"}\NormalTok{)}\OperatorTok{|}
\StringTok{                                                  }\KeywordTok{str_detect}\NormalTok{(location, }\StringTok{", TX"}\NormalTok{)}\OperatorTok{|}
\StringTok{                                                  }\KeywordTok{str_detect}\NormalTok{(location, }\StringTok{", UT"}\NormalTok{)}\OperatorTok{|}
\StringTok{                                                  }\KeywordTok{str_detect}\NormalTok{(location, }\StringTok{", VT"}\NormalTok{)}\OperatorTok{|}
\StringTok{                                                  }\KeywordTok{str_detect}\NormalTok{(location, }\StringTok{", VA"}\NormalTok{)}\OperatorTok{|}
\StringTok{                                                  }\KeywordTok{str_detect}\NormalTok{(location, }\StringTok{", WA"}\NormalTok{)}\OperatorTok{|}
\StringTok{                                                  }\KeywordTok{str_detect}\NormalTok{(location, }\StringTok{", WV"}\NormalTok{)}\OperatorTok{|}
\StringTok{                                                  }\KeywordTok{str_detect}\NormalTok{(location, }\StringTok{", WI"}\NormalTok{)}\OperatorTok{|}
\StringTok{                                                  }\KeywordTok{str_detect}\NormalTok{(location, }\StringTok{", WY"}\NormalTok{)}\OperatorTok{|}
\StringTok{                                                  }\KeywordTok{str_detect}\NormalTok{(location, }\StringTok{", DC"}\NormalTok{)}\OperatorTok{|}
\StringTok{                                                  }\KeywordTok{str_detect}\NormalTok{(location, }\StringTok{"Alabama"}\NormalTok{)}\OperatorTok{|}
\StringTok{                                                  }\KeywordTok{str_detect}\NormalTok{(location, }\StringTok{"Alaska"}\NormalTok{)}\OperatorTok{|}
\StringTok{                                                  }\KeywordTok{str_detect}\NormalTok{(location, }\StringTok{"Arizona"}\NormalTok{)}\OperatorTok{|}
\StringTok{                                                  }\KeywordTok{str_detect}\NormalTok{(location, }\StringTok{"Arkansas"}\NormalTok{)}\OperatorTok{|}
\StringTok{                                                  }\KeywordTok{str_detect}\NormalTok{(location, }\StringTok{"California"}\NormalTok{)}\OperatorTok{|}
\StringTok{                                                  }\KeywordTok{str_detect}\NormalTok{(location, }\StringTok{"Colorado"}\NormalTok{)}\OperatorTok{|}
\StringTok{                                                  }\KeywordTok{str_detect}\NormalTok{(location, }\StringTok{"Connecticut"}\NormalTok{)}\OperatorTok{|}
\StringTok{                                                  }\KeywordTok{str_detect}\NormalTok{(location, }\StringTok{"Delaware"}\NormalTok{)}\OperatorTok{|}
\StringTok{                                                  }\KeywordTok{str_detect}\NormalTok{(location, }\StringTok{"Florida"}\NormalTok{)}\OperatorTok{|}
\StringTok{                                                  }\KeywordTok{str_detect}\NormalTok{(location, }\StringTok{"Georgia"}\NormalTok{)}\OperatorTok{|}
\StringTok{                                                  }\KeywordTok{str_detect}\NormalTok{(location, }\StringTok{"Hawaii"}\NormalTok{)}\OperatorTok{|}
\StringTok{                                                  }\KeywordTok{str_detect}\NormalTok{(location, }\StringTok{"Idaho"}\NormalTok{)}\OperatorTok{|}
\StringTok{                                                  }\KeywordTok{str_detect}\NormalTok{(location, }\StringTok{"Illinois"}\NormalTok{)}\OperatorTok{|}
\StringTok{                                                  }\KeywordTok{str_detect}\NormalTok{(location, }\StringTok{"Indiana"}\NormalTok{)}\OperatorTok{|}
\StringTok{                                                  }\KeywordTok{str_detect}\NormalTok{(location, }\StringTok{"Iowa"}\NormalTok{)}\OperatorTok{|}
\StringTok{                                                  }\KeywordTok{str_detect}\NormalTok{(location, }\StringTok{"Kansas"}\NormalTok{)}\OperatorTok{|}
\StringTok{                                                  }\KeywordTok{str_detect}\NormalTok{(location, }\StringTok{"Kentucky"}\NormalTok{)}\OperatorTok{|}
\StringTok{                                                  }\KeywordTok{str_detect}\NormalTok{(location, }\StringTok{"Louisiana"}\NormalTok{)}\OperatorTok{|}
\StringTok{                                                  }\KeywordTok{str_detect}\NormalTok{(location, }\StringTok{"Maine"}\NormalTok{)}\OperatorTok{|}
\StringTok{                                                  }\KeywordTok{str_detect}\NormalTok{(location, }\StringTok{"Maryland"}\NormalTok{)}\OperatorTok{|}
\StringTok{                                                  }\KeywordTok{str_detect}\NormalTok{(location, }\StringTok{"Massachusetts"}\NormalTok{)}\OperatorTok{|}
\StringTok{                                                  }\KeywordTok{str_detect}\NormalTok{(location, }\StringTok{"Michigan"}\NormalTok{)}\OperatorTok{|}
\StringTok{                                                  }\KeywordTok{str_detect}\NormalTok{(location, }\StringTok{"Minnesota"}\NormalTok{)}\OperatorTok{|}
\StringTok{                                                  }\KeywordTok{str_detect}\NormalTok{(location, }\StringTok{"Mississippi"}\NormalTok{)}\OperatorTok{|}
\StringTok{                                                  }\KeywordTok{str_detect}\NormalTok{(location, }\StringTok{"Missouri"}\NormalTok{)}\OperatorTok{|}
\StringTok{                                                  }\KeywordTok{str_detect}\NormalTok{(location, }\StringTok{"Montana"}\NormalTok{)}\OperatorTok{|}
\StringTok{                                                  }\KeywordTok{str_detect}\NormalTok{(location, }\StringTok{"Nebraska"}\NormalTok{)}\OperatorTok{|}
\StringTok{                                                  }\KeywordTok{str_detect}\NormalTok{(location, }\StringTok{"Nevada"}\NormalTok{)}\OperatorTok{|}
\StringTok{                                                  }\KeywordTok{str_detect}\NormalTok{(location, }\StringTok{"New Hampshire"}\NormalTok{)}\OperatorTok{|}
\StringTok{                                                  }\KeywordTok{str_detect}\NormalTok{(location, }\StringTok{"New Jersey"}\NormalTok{)}\OperatorTok{|}
\StringTok{                                                  }\KeywordTok{str_detect}\NormalTok{(location, }\StringTok{"New Mexico"}\NormalTok{)}\OperatorTok{|}
\StringTok{                                                  }\KeywordTok{str_detect}\NormalTok{(location, }\StringTok{"New York"}\NormalTok{)}\OperatorTok{|}
\StringTok{                                                  }\KeywordTok{str_detect}\NormalTok{(location, }\StringTok{"North Carolina"}\NormalTok{)}\OperatorTok{|}
\StringTok{                                                  }\KeywordTok{str_detect}\NormalTok{(location, }\StringTok{"North Dakota"}\NormalTok{)}\OperatorTok{|}
\StringTok{                                                  }\KeywordTok{str_detect}\NormalTok{(location, }\StringTok{"Ohio"}\NormalTok{)}\OperatorTok{|}
\StringTok{                                                  }\KeywordTok{str_detect}\NormalTok{(location, }\StringTok{"Oklahoma"}\NormalTok{)}\OperatorTok{|}
\StringTok{                                                  }\KeywordTok{str_detect}\NormalTok{(location, }\StringTok{"Oregon"}\NormalTok{)}\OperatorTok{|}
\StringTok{                                                  }\KeywordTok{str_detect}\NormalTok{(location, }\StringTok{"Pennsylvania"}\NormalTok{)}\OperatorTok{|}
\StringTok{                                                  }\KeywordTok{str_detect}\NormalTok{(location, }\StringTok{"Rhode Island"}\NormalTok{)}\OperatorTok{|}
\StringTok{                                                  }\KeywordTok{str_detect}\NormalTok{(location, }\StringTok{"South Carolina"}\NormalTok{)}\OperatorTok{|}
\StringTok{                                                  }\KeywordTok{str_detect}\NormalTok{(location, }\StringTok{"South Dakota"}\NormalTok{)}\OperatorTok{|}
\StringTok{                                                  }\KeywordTok{str_detect}\NormalTok{(location, }\StringTok{"Tennessee"}\NormalTok{)}\OperatorTok{|}
\StringTok{                                                  }\KeywordTok{str_detect}\NormalTok{(location, }\StringTok{"Texas"}\NormalTok{)}\OperatorTok{|}
\StringTok{                                                  }\KeywordTok{str_detect}\NormalTok{(location, }\StringTok{"Utah"}\NormalTok{)}\OperatorTok{|}
\StringTok{                                                  }\KeywordTok{str_detect}\NormalTok{(location, }\StringTok{"Vermont"}\NormalTok{)}\OperatorTok{|}
\StringTok{                                                  }\KeywordTok{str_detect}\NormalTok{(location, }\StringTok{"Virginia"}\NormalTok{)}\OperatorTok{|}
\StringTok{                                                  }\KeywordTok{str_detect}\NormalTok{(location, }\StringTok{"Washington"}\NormalTok{)}\OperatorTok{|}
\StringTok{                                                  }\KeywordTok{str_detect}\NormalTok{(location, }\StringTok{"West Virginia"}\NormalTok{)}\OperatorTok{|}
\StringTok{                                                  }\KeywordTok{str_detect}\NormalTok{(location, }\StringTok{"Wisconsin"}\NormalTok{)}\OperatorTok{|}
\StringTok{                                                  }\KeywordTok{str_detect}\NormalTok{(location, }\StringTok{"Wyoming"}\NormalTok{)}
\NormalTok{)}

\CommentTok{#Obtenemos la cantidad de usuarios de nuestra muestra que no tweetean desde Estados Unidos}
\NormalTok{n_extranjeros <-}\StringTok{ }\NormalTok{n_muestra}\OperatorTok{-}\KeywordTok{length}\NormalTok{(muestra_followers}\OperatorTok{$}\NormalTok{user_id)}


\CommentTok{#Estimamos la proporción de de tweets extranjeros}
\NormalTok{estim         <-}\StringTok{ }\NormalTok{n_extranjeros}\OperatorTok{/}\NormalTok{n_muestra}

\CommentTok{#A partir de esta estimación puntual de la proporción de followers extranjeros de Trump}
\CommentTok{#crearemos un intervalo de confianza a nivel 99%}
\NormalTok{N                 <-}\StringTok{ }\DecValTok{83650000} \CommentTok{#Numero total de followers de Trump }
\NormalTok{alpha             <-}\StringTok{ }\FloatTok{0.01}
\NormalTok{Z                 <-}\StringTok{ }\KeywordTok{qnorm}\NormalTok{(}\DecValTok{1} \OperatorTok{-}\StringTok{ }\NormalTok{alpha}\OperatorTok{/}\DecValTok{2}\NormalTok{)}
\NormalTok{varianza.gorro    <-}\StringTok{ }\NormalTok{estim}\OperatorTok{*}\NormalTok{(}\DecValTok{1}\OperatorTok{-}\NormalTok{estim)}\OperatorTok{*}\NormalTok{(N }\OperatorTok{-}\StringTok{ }\NormalTok{n_muestra)}\OperatorTok{/}\NormalTok{n_muestra}\OperatorTok{/}\NormalTok{(N}\DecValTok{-1}\NormalTok{)}
\NormalTok{limite.inferior   <-}\StringTok{ }\NormalTok{estim }\OperatorTok{-}\StringTok{ }\NormalTok{Z}\OperatorTok{*}\KeywordTok{sqrt}\NormalTok{(varianza.gorro)}
\NormalTok{limite.superior   <-}\StringTok{ }\NormalTok{estim }\OperatorTok{+}\StringTok{ }\NormalTok{Z}\OperatorTok{*}\KeywordTok{sqrt}\NormalTok{(varianza.gorro)}
\KeywordTok{paste0}\NormalTok{(}\StringTok{"IC: ["}\NormalTok{,}
\NormalTok{       limite.inferior, }\StringTok{","}\NormalTok{, limite.superior,}\StringTok{"]"}\NormalTok{)}
\end{Highlighting}
\end{Shaded}

\begin{verbatim}
## [1] "IC: [0.656980980887364,0.727770792162282]"
\end{verbatim}

\hypertarget{conclusiuxf3n}{%
\subsection{Conclusión}\label{conclusiuxf3n}}

La opinión de la gente es algo sumamente relevante en algunos temas y
simplemente es difícil de ignorar. A través del análisis de sentimiento
y la minería de datos en redes se puede construir una idea bastante
fidedigna de la percepción que está teniendo la gente sobre una marca,
producto, negocio, tema polémico o persona. Esto no es poca cosa ya que
el tener esta posibilidad, podemos utilizar la información para tomar
mejores decisiones sobre estrategias políticas o de negocio.

\hypertarget{referencias}{%
\subsection{Referencias}\label{referencias}}

\begin{enumerate}
\def\labelenumi{\arabic{enumi}.}
\tightlist
\item
  Repositorio de JBGruber ** recover\_stream.R**. Link:
  \url{https://gist.github.com/JBGruber/dee4c44e7d38d537426f57ba1e4f84ab}
\item
  Lohr, S. L. (2019). Sampling design and analysis. Boca Raton, FL: CRC
  Press.
\end{enumerate}

\end{document}
